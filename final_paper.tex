% Options for packages loaded elsewhere
\PassOptionsToPackage{unicode}{hyperref}
\PassOptionsToPackage{hyphens}{url}
\PassOptionsToPackage{dvipsnames,svgnames,x11names}{xcolor}
%
\documentclass[
  letterpaper,
  DIV=11,
  numbers=noendperiod]{scrartcl}

\usepackage{amsmath,amssymb}
\usepackage{iftex}
\ifPDFTeX
  \usepackage[T1]{fontenc}
  \usepackage[utf8]{inputenc}
  \usepackage{textcomp} % provide euro and other symbols
\else % if luatex or xetex
  \usepackage{unicode-math}
  \defaultfontfeatures{Scale=MatchLowercase}
  \defaultfontfeatures[\rmfamily]{Ligatures=TeX,Scale=1}
\fi
\usepackage{lmodern}
\ifPDFTeX\else  
    % xetex/luatex font selection
\fi
% Use upquote if available, for straight quotes in verbatim environments
\IfFileExists{upquote.sty}{\usepackage{upquote}}{}
\IfFileExists{microtype.sty}{% use microtype if available
  \usepackage[]{microtype}
  \UseMicrotypeSet[protrusion]{basicmath} % disable protrusion for tt fonts
}{}
\makeatletter
\@ifundefined{KOMAClassName}{% if non-KOMA class
  \IfFileExists{parskip.sty}{%
    \usepackage{parskip}
  }{% else
    \setlength{\parindent}{0pt}
    \setlength{\parskip}{6pt plus 2pt minus 1pt}}
}{% if KOMA class
  \KOMAoptions{parskip=half}}
\makeatother
\usepackage{xcolor}
\setlength{\emergencystretch}{3em} % prevent overfull lines
\setcounter{secnumdepth}{3}
% Make \paragraph and \subparagraph free-standing
\ifx\paragraph\undefined\else
  \let\oldparagraph\paragraph
  \renewcommand{\paragraph}[1]{\oldparagraph{#1}\mbox{}}
\fi
\ifx\subparagraph\undefined\else
  \let\oldsubparagraph\subparagraph
  \renewcommand{\subparagraph}[1]{\oldsubparagraph{#1}\mbox{}}
\fi


\providecommand{\tightlist}{%
  \setlength{\itemsep}{0pt}\setlength{\parskip}{0pt}}\usepackage{longtable,booktabs,array}
\usepackage{calc} % for calculating minipage widths
% Correct order of tables after \paragraph or \subparagraph
\usepackage{etoolbox}
\makeatletter
\patchcmd\longtable{\par}{\if@noskipsec\mbox{}\fi\par}{}{}
\makeatother
% Allow footnotes in longtable head/foot
\IfFileExists{footnotehyper.sty}{\usepackage{footnotehyper}}{\usepackage{footnote}}
\makesavenoteenv{longtable}
\usepackage{graphicx}
\makeatletter
\def\maxwidth{\ifdim\Gin@nat@width>\linewidth\linewidth\else\Gin@nat@width\fi}
\def\maxheight{\ifdim\Gin@nat@height>\textheight\textheight\else\Gin@nat@height\fi}
\makeatother
% Scale images if necessary, so that they will not overflow the page
% margins by default, and it is still possible to overwrite the defaults
% using explicit options in \includegraphics[width, height, ...]{}
\setkeys{Gin}{width=\maxwidth,height=\maxheight,keepaspectratio}
% Set default figure placement to htbp
\makeatletter
\def\fps@figure{htbp}
\makeatother

\newcolumntype{d}{S[input-symbols = ()]}
\usepackage{rotating}
\KOMAoption{captions}{tableheading}
\makeatletter
\@ifpackageloaded{caption}{}{\usepackage{caption}}
\AtBeginDocument{%
\ifdefined\contentsname
  \renewcommand*\contentsname{Table of contents}
\else
  \newcommand\contentsname{Table of contents}
\fi
\ifdefined\listfigurename
  \renewcommand*\listfigurename{List of Figures}
\else
  \newcommand\listfigurename{List of Figures}
\fi
\ifdefined\listtablename
  \renewcommand*\listtablename{List of Tables}
\else
  \newcommand\listtablename{List of Tables}
\fi
\ifdefined\figurename
  \renewcommand*\figurename{Figure}
\else
  \newcommand\figurename{Figure}
\fi
\ifdefined\tablename
  \renewcommand*\tablename{Table}
\else
  \newcommand\tablename{Table}
\fi
}
\@ifpackageloaded{float}{}{\usepackage{float}}
\floatstyle{ruled}
\@ifundefined{c@chapter}{\newfloat{codelisting}{h}{lop}}{\newfloat{codelisting}{h}{lop}[chapter]}
\floatname{codelisting}{Listing}
\newcommand*\listoflistings{\listof{codelisting}{List of Listings}}
\makeatother
\makeatletter
\makeatother
\makeatletter
\@ifpackageloaded{caption}{}{\usepackage{caption}}
\@ifpackageloaded{subcaption}{}{\usepackage{subcaption}}
\makeatother
\ifLuaTeX
  \usepackage{selnolig}  % disable illegal ligatures
\fi
\usepackage{bookmark}

\IfFileExists{xurl.sty}{\usepackage{xurl}}{} % add URL line breaks if available
\urlstyle{same} % disable monospaced font for URLs
\hypersetup{
  pdfauthor={Group C: Chaira, Simi, Melanie, Ellie},
  colorlinks=true,
  linkcolor={blue},
  filecolor={Maroon},
  citecolor={Blue},
  urlcolor={Blue},
  pdfcreator={LaTeX via pandoc}}

\title{ECO285 Case 7:\\
Research Affiliates and Dynamic Multifactor Strategies}
\author{Group C: Chaira, Simi, Melanie, Ellie}
\date{}

\begin{document}
\maketitle

Several pivotal decisions face the Research Affiliates team regarding
the future of the strategy underlying the PIMCO RAFI Multifactor ETF
(MFUS). MFUS delivered an aggregate return of 10.02\% from December 2017
to February 2021, underperforming the broader market's return of 14.56\%
by 4.54\%, as indexed by the Russell 1000. MFUS also trailed competing
ETFs that employed the same dynamic, multifactor strategy, such as the
Invesco Russell 1000 Dynamic Multifactor ETF (OMFL), which returned
18.4\% in the same period. This underperformance to both the benchmark
and competitors prompted a review of three key elements of Research
Affiliates' strategy: factor selection, the dynamic allocation
methodology, and the use of fundamental weighting.

Research Affiliates selected five smart-beta strategies with low and
negative correlations for MFUS: value, low volatility, quality,
momentum, and size. From 1968 to 2016, these factors averaged an annual
return of 13.14\% (Exhibit 13). Beyond historical performance,
valuation-based predictability supported their inclusion. For example,
the RAFI Fundamental Index, which weights companies using four
fundamental value measures, shows a negative relationship between
price-to-book (P/B) ratio and future five-year returns, suggesting that
cheaper strategies tend to produce higher future returns (Exhibit 1).
The Low Volatility Index similarly demonstrates a decline in returns as
the P/B ratio increases, with expected 5-year returns falling from
around 18\% to -13\% as the P/B ratio increases from around 0.3 to 0.8,
suggesting that the strategy is sensitive to relative valuation, making
it better suited for timing in the dynamic strategy (Exhibit 1).

In contrast, the Momentum factor demonstrates a weaker, more dispersed
relationship. Returns are clustered in the 0-10\% range across a broader
range of P/B ratios (1-5) relative to Volatility, which suggests that
timing offers limited predictability for this factor, based on
valuation. Despite this, momentum received high allocations in practice,
ranging from 18\% to 35\%, the upper limit, which suggests its z-score
was driven more by recent outperformance than by valuation reversion.
The divergence in responsiveness of different factors to valuation-based
timing is a key limitation in Research Affiliates' strategy. While the
low-volatility factor responds well to valuation-based timing, other
factors, such as momentum, are better captured through other measures
that are not accounted for in the allocation strategy.

MFUS's dynamic allocation methodology adjusts weights based on two
timing signals: factor reversal (performance over the last 5 years) and
factor momentum (return over the previous year). These signals are
standardized into z-scores, which then determine shifts in the dynamic
allocation. From 2016 to 2020, the Low Volatility factor saw the widest
allocation range of the five factors, ranging from around 8\% to 30\%.
This wide range reflects the sensitivity of the volatility factor to
valuation, where returns sharply decline as P/B ratios increase,
supporting significant allocation shifts as valuations rise. The
Momentum allocation shows a narrower allocation range from around 18\%
to 35\%, the upper limit. Momentum's high, relatively stable allocation
is driven more by strong recent performance than valuation reversals, as
Exhibit 1 shows the Momentum factor displays a weak, more dispersed
relationship between the P/B ratio and future returns, which shows
clustering around the 10\% return ratio regardless of valuation.

This relationship suggests that valuation-based timing is less effective
for the Momentum factor relative to factors with a stronger
valuation/return relationship, suggesting its allocation shifts were
more likely to be driven by positive-momentum z-scores. This uneven
responsiveness across valuation-sensitive factors is a key limitation of
the dynamic allocation strategy, where it performs better for
valuation-sensitive factors but is less effective for trend-based
factors like momentum, which persist despite appearing to be more
expensive.

MFUS utilizes a fundamental weighting strategy rather than the more
commonly used market capitalization approach, aiming to anchor portfolio
weights in long-term business fundamentals, rather than market
sentiment. The strategy determined company weights by averaging four
accounting-based measures: de-levered sales, operating cash flow,
average dividends paid and share buybacks, and most recent book value.
These variables favor companies with strong and consistent fundamentals
historically associated with future returns, which is necessary in the
strategy that aims to use value-tilted factors to reduce exposure to
overvalued stocks. This valuation strategy seeks to offer protection
during speculative periods, such as the late 1990s tech bubble, which
saw overpriced growth stocks put a drag on returns.

However, the value-tilted design created a value bias that limited
MFUS's exposure to growth-driven companies. MFUS allocated 61\% to
large-cap stocks and just 10\% to small-caps, while competing ETF OMFL
allocated 61\% of its portfolio to mid-cap stocks. Sector composition
within MFUS skewed towards sensitive sectors (40.72\% of allocation),
namely technology (18.09\%) and industrials (11.4\%), while OMFL weighed
heavily in cyclical sectors (53.46\%), specifically financials (24.14\%)
and consumer cyclical companies (19.46\%). These sector tilts allowed
OMFL to reap more rewards from the economic expansion that defined much
of the period from 2017 to early 2020, while MFUS's value tilt lagged.

Exhibit 12 shows that MFUS had top portfolio weightings in large, stable
companies, averaging P/B ratios of \$3.34, such as Apple (2.62\%),
Walmart (1.72\%), and Home Depot (1.38\%), consistent with the
fundamental weighting strategy. OMFL held higher weights in mid-cap,
more economically-sensitive firms that averaged P/B ratios of \$3.05,
such as ViacomCBS (1.06\%) and United Rentals (0.92\%) that benefited
from cyclical growth. The divergence in portfolio design is seen in
portfolio turnover, where OMFL's turnover ratio of 321\% indicates
high-frequency rebalancing, while MFUS's 36\% turnover ratio is more
consistent with its long-term, valuation focus and reliance on more
slowly evolving fundamental metrics.

The PIMCO RAFI Dynamic Multifactor ETF (MFUS) implements a dynamic
weighting strategy designed by Research Affiliates, aiming to outperform
static multifactor models by tactically allocating more capital to
factors that appear cheap or have performed well recently. This
contrasts with ETFs like Invesco's OMFL, which also uses a multifactor
framework but relies more heavily on macroeconomic regime-based factor
tilts.

The primary strength of dynamic weighting is its theoretical ability to
improve returns by adapting factor exposures based on changing market
conditions. This approach recognizes that factors are cyclical. Value
and Size, for instance, may underperform for several years before
experiencing rebounds. By overweighting factors that have underperformed
over a 4-year horizon but shown recent 12-month strength, MFUS seeks to
enter those rebounds early. This approach also addresses a common
critique of static models which is that they may remain overexposed to
expensive or overcrowded factors.

Another benefit is flexibility. A dynamic model can reduce exposure to
factors that are expensive or exhibiting negative momentum, potentially
avoiding extended periods of underperformance. This ability to tilt away
from crowded trades or overvalued factors could help improve
risk-adjusted returns compared to static equal-weighted approaches.

Despite its appeal, dynamic weighting is not without risks. It depends
heavily on the reliability of the timing signals used. These signals can
fail, especially during economic shifts. From 2017 to 2021, MFUS
significantly underperformed both the Russell 1000 and competitor
multifactor ETFs like OMFL, calling into question the effectiveness of
its timing approach. This underperformance suggests that even
well-designed dynamic models may be prone to mis-timing or may
overweight structurally underperforming factors like Value, especially
when traditional valuation metrics do not capture intangible value well.

OMFL (Invesco) offers a helpful comparison. While it also rotates factor
exposures, it does so based on a macroeconomic regime model, adjusting
exposure to factors like Low Volatility or Momentum based on growth,
inflation, and risk conditions. Its approach has delivered stronger
recent returns, suggesting that macro-based timing may have been more
effective than valuation/momentum-based timing during this period.



\end{document}
